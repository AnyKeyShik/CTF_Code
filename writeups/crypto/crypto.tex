%%%%%%%%%%%%%%%%%%%%%%%%%%%%%%%%%%%%%%%%%
% Daily Laboratory Book
% LaTeX Template
% Version 1.0 (4/4/12)
%
% This template has been downloaded from:
% http://www.LaTeXTemplates.com
%
% Original author:
% Frank Kuster (http://www.ctan.org/tex-archive/macros/latex/contrib/labbook/)
%
% Important note:
% This template requires the labbook.cls file to be in the same directory as the
% .tex file. The labbook.cls file provides the necessary structure to create the
% lab book.
%
% The \lipsum[#] commands throughout this template generate dummy text
% to fill the template out. These commands should all be removed when 
% writing lab book content.
%
% HOW TO USE THIS TEMPLATE 
% Each day in the lab consists of three main things:
%
% 1. LABDAY: The first thing to put is the \labday{} command with a date in 
% curly brackets, this will make a new page and put the date in big letters 
% at the top.
%
% 2. EXPERIMENT: Next you need to specify what experiment(s) you are 
% working on with an \experiment{} command with the experiment shorthand 
% in the curly brackets. The experiment shorthand is defined in the 
% 'DEFINITION OF EXPERIMENTS' section below, this means you can 
% say \experiment{pcr} and the actual text written to the PDF will be what 
% you set the 'pcr' experiment to be. If the experiment is a one off, you can 
% just write it in the bracket without creating a shorthand. Note: if you don't 
% want to have an experiment, just leave this out and it won't be printed.
%
% 3. CONTENT: Following the experiment is the content, i.e. what progress 
% you made on the experiment that day.
%
%%%%%%%%%%%%%%%%%%%%%%%%%%%%%%%%%%%%%%%%%

%----------------------------------------------------------------------------------------
%	PACKAGES AND OTHER DOCUMENT CONFIGURATIONS
%----------------------------------------------------------------------------------------

\documentclass[idxtotoc,hyperref,openany,oneside]{files/crypto} % 'openany' here removes the gap page between days, erase it to restore this gap; 'oneside' can also be added to remove the shift that odd pages have to the right for easier reading

\usepackage[ 
  backref=page,
  pdfpagelabels=true,
  plainpages=false,
  colorlinks=true,
  bookmarks=true,
  pdfview=FitB]{hyperref} % Required for the hyperlinks within the PDF
  
\usepackage{booktabs} % Required for the top and bottom rules in the table
\usepackage{float} % Required for specifying the exact location of a figure or table
\usepackage{graphicx} % Required for including images2
\usepackage{listings} % Used for programs' listings

\usepackage[english,russian]{babel}
\usepackage[utf8]{inputenc}
\usepackage [T2A] {fontenc}

\newcommand{\HRule}{\rule{\linewidth}{0.5mm}} % Command to make the lines in the title page
\setlength\parindent{0pt} % Removes all indentation from paragraphs

%----------------------------------------------------------------------------------------
%	DEFINITION OF EXPERIMENTS
%----------------------------------------------------------------------------------------

\newexperiment{easy1}{Based task}
\newexperiment{easy2}{Hashes among us}
\newexperiment{medium1}{Please be careful with ASR}
\newexperiment{medium2}{Please don't share}
\newexperiment{hard1}{Do you want to play some gamel?}
\newexperiment{hard2}{Swift task}

%---------------------------------------------------------------------------------------

\begin{document}

%----------------------------------------------------------------------------------------
%	TITLE PAGE
%----------------------------------------------------------------------------------------

\frontmatter % Use Roman numerals for page numbers
\title{
\begin{center}
\HRule \\[0.4cm]
{\Huge \bfseries CTF Code \\[0.5cm] \Large Writeups}\\[0.4cm] % Degree
\HRule \\[1.5cm]
\end{center}
}
\author{\Huge Криптография \\ \\[2cm]} % Your name and email address
\maketitle

\tableofcontents

\mainmatter % Use Arabic numerals for page numbers

%----------------------------------------------------------------------------------------
%	LAB BOOK CONTENTS
%----------------------------------------------------------------------------------------

% Blank template to use for new days:

%\labday{Day, Date Month Year}

%\experiment{}

%Text

%-----------------------------------------

%\experiment{}

%Text

%----------------------------------------------------------------------------------------

\labday{Easy}

\experiment{easy1}

Достаточно простая задача. Нам дается файл с какими-то иероглифами:
\begin{figure}[H]
\begin{center}
\includegraphics[width=1.0\linewidth]{files/chinese}
\end{center}
\caption{Китайцы уже близко}
\label{fig:chinese}
\end{figure}
Казалось бы, что тут можно придумать, переводчик выдает какую-то дичь. Если заметить название таска, то можно подумать про какую-то кодировку из base'ов. Немного гуглинга и можно наткнуться на весьма интересную штуку под названием \href{https://github.com/qntm/base65536}{base65536}. Прогнав через него натыкаемся на какую-то случайную последовательность эмодзи:
\begin{figure}[H]
\begin{center}
\includegraphics[width=1.0\linewidth]{files/emoji}
\end{center}
\caption{Кто-то слишком эмоционален}
\label{fig:emoji}
\end{figure}
Погуглив еще немного можно найти \href{https://github.com/AdamNiederer/base100}{base100}, после извлечения из которого получаем старый-добрый base64:
\begin{figure}[H]
\begin{center}
\includegraphics[width=0.7\linewidth]{files/base64}
\end{center}
\caption{То, что знакомо почти всем}
\label{fig:base64}
\end{figure}
И тут два варианта:
\begin{itemize}
\item Прогнать обратно ручками
\item Написать питоновский скрипт
\end{itemize}
Чтобы райтап был полным, рассмотрим второй вариант, потому что первый достаточно очевидный и не требует пояснений. Скрипт для расшифровки выглядит примерно следующим образом:
\newpage
\begin{lstlisting}[language=Python, caption=Дешифровка флага]
#! /usr/bin/env python3
# -*- coding: utf-8 -*-

import base65536
import pybase100 as base100
import base64

with open('flag.enc', 'r') as flag_file:
    flag = flag_file.read()

flag = base65536.decode(flag)
flag = base100.decode(flag)
flag = base64.b64decode(flag)
flag = flag.decode('utf-8')

print(flag)
\end{lstlisting}

На выходе получаем флаг \verb|oren_ctf_KevinDavidMitnick!|.

%-----------------------------------------

\experiment{easy2}

Судя по виду зашифрованного флага и названию задачи перед нами какой-то хэш.
\begin{figure}[H]
\begin{center}
\includegraphics[width=1.0\linewidth]{files/md5flag}
\end{center}
\caption{Слишком много хексов}
\label{fig:chinese}
\end{figure}
Если посмотреть длинну, то мы увидим, что длинна флага ровно 32 символа, что позволяет подумать про MD5. Дальше можно либо брутить локально с помощью HashCat/JhonTheRipper или воспользоваться онлайн-сервисами наподобие \href{https://crackstation.net/}{CrackStation}. После чего получаем строку \verb|RobertMorris|, которую нужно обернуть в \verb|oren_ctf_| и \verb|!|. После чего получаем флаг \verb|oren_ctf_RobertMorris!|.

%----------------------------------------------------------------------------------------

\labday{Medium}

\experiment{medium1}

Судя по ключам и названию задчи речь явно идет про RSA. Что же, можно погуглить про атаки на этот алгоритм и наткнуться на весьма интересную штуку под названием \href{https://en.wikipedia.org/wiki/Coppersmith\%27s_attack#H\%C3\%A5stad's_broadcast_attack}{Håstad's broadcast attack}. После чего, если попробовать сдампить открытую экспоненту и модуль из наших открытых ключей, то можно увидеть, что во всех трех ключах экспонента маленькая и равна 3. Что уже точно намекает на эту атаку. 

\textbf{Суть атаки}

Пользователь $А$ отсылает зашифрованное сообщение m нескольким пользователям. в данном случае, трём (по числу файлов): $P1$, $P2$, $P3$. У каждого пользователя есть свой ключ, представляемый парой «модуль-открытая экспонента» ($n_i$, $e_i$), причём $M < n1$, $n2$, $n3$. Для каждого из трех пользователей $А$ зашифровывает сообщение на соответствующем открытом ключе и отсылает результат адресату.
Атакующий же реализует перехват сообщений и собирает переданные шифртексты (обозначим их как $C_1$, $C_2$, $C_3$), с целью восстановить исходное сообщение $M$. Значит, по имеющимся трем шифртекстам нужно восстановить сообщение, которое будет флагом.

\textbf{Почему точно сможем ее реализовать?}

Как известно, шифрование сообщения по схеме RSA происходит следующим образом: $C = M^e \pmod{n}$. В случае с открытой экспонентой, равной 3, получение шифртекстов выглядит так:
\begin{center}
$C_1 = M^3 \pmod{n_1}$

$C_2 = M^3 \pmod{n_2}$

$C_3 = M^3 \pmod{n_3}$
\end{center}
Зная, что $n_1$, $n_2$, $n_3$ взаимно просты, можем применить к шифртекстам \href{https://ru.wikipedia.org/wiki/\%D0\%9A\%D0\%B8\%D1\%82\%D0\%B0\%D0\%B9\%D1\%81\%D0\%BA\%D0\%B0\%D1\%8F_\%D1\%82\%D0\%B5\%D0\%BE\%D1\%80\%D0\%B5\%D0\%BC\%D0\%B0_\%D0\%BE\%D0\%B1_\%D0\%BE\%D1\%81\%D1\%82\%D0\%B0\%D1\%82\%D0\%BA\%D0\%B0\%D1\%85}{китайскую теорему об остатках}. Получим в итоге некоторое $C'$, корень кубический из которого и даст нам искомое сообщение $M$.
\begin{center}
$C' = M^3 \pmod{n_1*n_2*n_3}$
\end{center}
Вспоминаем, что $M$ меньше каждого из трёх модулей $n_i$, значит, справедливо равенство:
\begin{center}
$C' = M^3$
\end{center}
Так мы и найдем наше искомое сообщение $M$.

\newpage
\textbf{Программная реализация атаки Хастада}
После всего вышесказанного не составляет труда написать простенький скрипт на питоне, который выдает зашифрованное сообщение:
\begin{lstlisting}[language=Python, caption=Атака Хастада]
#!/usr/bin/env python3
# -*- coding: utf-8 -*-

import gmpy2
gmpy2.get_context().precision = 2048

from binascii import unhexlify
from functools import reduce
from gmpy2 import root
from Crypto.PublicKey import RSA

def chinese_remainder_theorem(items):
    N = 1
    for a, n in items:
        N *= n

    result = 0
    for a, n in items:
        m = N // n
        r, s, d = extended_gcd(n, m)
        if d != 1:
            raise "Input not pairwise co-prime"
        result += a * s * m

    return result % N


def extended_gcd(a, b):
    x, y = 0, 1
    lastx, lasty = 1, 0

    while b:
        a, (q, b) = b, divmod(a, b)
        x, lastx = lastx - q * x, x
        y, lasty = lasty - q * y, y

    return (lastx, lasty, a)


def mul_inv(a, b):
    b0 = b
    x0, x1 = 0, 1
    if b == 1:
        return 1
    while a > 1:
        q = a // b
        a, b = b, a % b
        x0, x1 = x1 - q * x0, x0
    if x1 < 0:
        x1 += b0

    return x1

def get_cipher(filename):
    with open(filename, 'rb') as cipher:
        value = cipher.read().hex()

    return int(value, 16)

def get_modulus(filename):
    with open(filename) as keyfile:
        keystr = keyfile.read()
        key = RSA.import_key(keystr)

    return key.n

if __name__ == '__main__':

    ciphertext1 = get_cipher("flag.enc.alice")
    ciphertext2 = get_cipher("flag.enc.bob")
    ciphertext3 = get_cipher("flag.enc.eve")

    modulus1 = get_modulus("alice.pub")
    modulus2 = get_modulus("bob.pub")
    modulus3 = get_modulus("eve.pub")

    C = chinese_remainder_theorem([(ciphertext1, modulus1), (ciphertext2, modulus2), (ciphertext3, modulus3)])
    flag = int(root(C, 3))
    flag = hex(flag)[2:]

    print(unhexlify(flag).decode('utf-8'), end='')
\end{lstlisting}

После выполнения скрипта на выходе получаем расшифрованный текст с флагом: \verb|CIH, also known as Chernobyl or Spacefiller, is a Microsoft Windows 9x computer virus which first emerged in 1998. Its payload is overwriting critical information on infected system drives, and in some cases destroying the system BIOS. Flag: oren_ctf_CIH!|


%-----------------------------------------


\experiment{medium2}

При подключении к серверу мы видим сообщение
\begin{figure}[H]
\begin{center}
\includegraphics[width=1.0\linewidth]{files/shamir_aes}
\end{center}
\caption{Это "жжж" явно неспроста}
\label{fig:chinese}
\end{figure}
Исходя из названия задачи и формулировки подсказки можно предположить, что речь идет о \href{https://en.wikipedia.org/wiki/Shamir\%27s_Secret_Sharing}{Shamir Secret Sharing Scheme} (или иногда можно встретить сокращение SSSS). Помимо этого есть опечатка в слове \verb|event|, что тоже дает какую-то подсказку. По подсказке понятно, что схемой разбит не сам флаг, а ключ для какого-то алгоримта шифрования с нулевым вектором инициализации, которым уже зашифрован флаг.

\textbf{Немного о схеме Шамира}

Если очень кратко, то секрет (информация, которой мы хотим поделиться) $S$ разбивается на $n$ частей таким образом, чтобы можно было восстановить исходное сообщение $S$ только в случае, если мы имеем не менее $k$ кусочков из $n$. Даже наличие $k-1$ куска недостаточно, чтобы восстановить исходную $S$. Чуть более подробно про это написано вот \href{https://habr.com/ru/post/431392/}{здесь}.

\textbf{Реализация}

Можно писать разделение секрета самому, но это выходит за рамки данного райтапа. Если немного погуглить, то можно найти достаточно большое количество реализаций этого алгоритма на самых разных языках. Для питона существует вот \href{https://github.com/shea256/secret-sharing}{эта} реализация, но, к сожалению, она имеет проблемы с третьей версией из-за использования типа \verb|long|, который был удален. Но ничего не мешает написать скрипт на втором. Итак, после получения полного секрета можно заметить, что это строка длинной 16 символов, что, в купе с нулевым инициализирующим вектором, дает возможность предположить, что используется AES-128. Остается лишь понять, какой из двух самых известных вариантов используется - ECB или CBC. Это можно сделать просто попробовав каждый из них. После чего можно написать небольшой питоновский скрипт, чтобы получить флаг:
\newpage
\begin{lstlisting}[language=Python, caption=Расшифровка флага AES-CBC]
#!/usr/bin/env python2
# -*- coding: utf-8 -*-

import sys
from base64 import b64decode
from Crypto.Cipher import AES
from secretsharing import SecretSharer


if len(sys.argv) < 4:
    print("Usage: {} part1 part2 part2".format(sys.argv[0]))
    exit(1)

AES_KEY = SecretSharer.recover_secret([sys.argv[1], sys.argv[2], sys.argv[3]])
IV = b'\x00' * 16
cipher = AES.new(key=AES_KEY, mode=AES.MODE_CBC, iv=IV)

with open('flag.enc', 'r') as flagfile:
    enc_flag = b64decode(flagfile.read())

flag = cipher.decrypt(enc_flag)

print("Decrypted flag is: {}".format(flag))
\end{lstlisting}


%----------------------------------------------------------------------------------------

\labday{Hard}

\experiment{hard1}

%-----------------------------------------


\experiment{hard2}

%----------------------------------------------------------------------------------------

\end{document}