%%%%%%%%%%%%%%%%%%%%%%%%%%%%%%%%%%%%%%%%%
% Daily Laboratory Book
% LaTeX Template
% Version 1.0 (4/4/12)
%
% This template has been downloaded from:
% http://www.LaTeXTemplates.com
%
% Original author:
% Frank Kuster (http://www.ctan.org/tex-archive/macros/latex/contrib/labbook/)
%
% Important note:
% This template requires the labbook.cls file to be in the same directory as the
% .tex file. The labbook.cls file provides the necessary structure to create the
% lab book.
%
% The \lipsum[#] commands throughout this template generate dummy text
% to fill the template out. These commands should all be removed when 
% writing lab book content.
%
% HOW TO USE THIS TEMPLATE 
% Each day in the lab consists of three main things:
%
% 1. LABDAY: The first thing to put is the \labday{} command with a date in 
% curly brackets, this will make a new page and put the date in big letters 
% at the top.
%
% 2. EXPERIMENT: Next you need to specify what experiment(s) you are 
% working on with an \experiment{} command with the experiment shorthand 
% in the curly brackets. The experiment shorthand is defined in the 
% 'DEFINITION OF EXPERIMENTS' section below, this means you can 
% say \experiment{pcr} and the actual text written to the PDF will be what 
% you set the 'pcr' experiment to be. If the experiment is a one off, you can 
% just write it in the bracket without creating a shorthand. Note: if you don't 
% want to have an experiment, just leave this out and it won't be printed.
%
% 3. CONTENT: Following the experiment is the content, i.e. what progress 
% you made on the experiment that day.
%
%%%%%%%%%%%%%%%%%%%%%%%%%%%%%%%%%%%%%%%%%

%----------------------------------------------------------------------------------------
%	PACKAGES AND OTHER DOCUMENT CONFIGURATIONS
%----------------------------------------------------------------------------------------

\documentclass[idxtotoc,hyperref,openany,oneside]{files/reverse} % 'openany' here removes the gap page between days, erase it to restore this gap; 'oneside' can also be added to remove the shift that odd pages have to the right for easier reading

\usepackage[ 
  backref=page,
  pdfpagelabels=true,
  plainpages=false,
  colorlinks=true,
  bookmarks=true,
  pdfview=FitB]{hyperref} % Required for the hyperlinks within the PDF
  
\usepackage{booktabs} % Required for the top and bottom rules in the table
\usepackage{float} % Required for specifying the exact location of a figure or table
\usepackage{graphicx} % Required for including images2
\usepackage{listings} % Used for programs' listings
\usepackage{tcolorbox} % For textboxes

\usepackage[english,russian]{babel}
\usepackage[utf8]{inputenc}
\usepackage[T2A]{fontenc}

\newcommand{\HRule}{\rule{\linewidth}{0.5mm}} % Command to make the lines in the title page
\setlength\parindent{0pt} % Removes all indentation from paragraphs

%----------------------------------------------------------------------------------------
%	DEFINITION OF EXPERIMENTS
%----------------------------------------------------------------------------------------

\newexperiment{easy1}{Check the license!}
\newexperiment{easy2}{Guess the password}
\newexperiment{medium}{<Название>}
\newexperiment{hard1}{<Название>}
\newexperiment{hard2}{<Название>}

%---------------------------------------------------------------------------------------

\begin{document}

%----------------------------------------------------------------------------------------
%	TITLE PAGE
%----------------------------------------------------------------------------------------

\frontmatter % Use Roman numerals for page numbers
\title{
\begin{center}
\HRule \\[0.4cm]
{\Huge \bfseries CTF Code \\[0.5cm] \Large Writeups}\\[0.4cm] % Degree
\HRule \\[1.5cm]
\end{center}
}
\author{\Huge Reverse Engineering \\ \\[2cm]} % Your name and email address
\maketitle

\tableofcontents

\mainmatter % Use Arabic numerals for page numbers

%----------------------------------------------------------------------------------------
%	LAB BOOK CONTENTS
%----------------------------------------------------------------------------------------

% Blank template to use for new days:

%\labday{Day, Date Month Year}

%\experiment{}

%Text

%-----------------------------------------

%\experiment{}

%Text

%----------------------------------------------------------------------------------------

\labday{Easy}

\experiment{easy1}

\textbf{Теги:} Java, License key\vspace{\baselineskip}

\begin{tcolorbox}
<условие задачи>
\end{tcolorbox}

Нам дается программа на Java, которая хочет какую-то лицензию. Самое время ее разреверсить и посмотреть, что же там за лицензия нам нужна. Так как это Java, то можно восстановить исходный код с точностью до имен переменных с помощью любого декомпилятора. В райтапе будет использоваться JD-GUI. После открытия файла видим, что он совсем небольшой и состоит всего из трех классов:
\begin{figure}[H]
\begin{center}
\includegraphics[width=0.7\linewidth]{files/java-classes}
\end{center}
\caption{Java изнутри}
\label{fig:java-classes}
\end{figure}

После рассмотрения \verb|Main|'a понимаем, что это просто драйвер и ничего связанного с лицензией или ее обработкой не делает. С классом \verb|LicenseHandler| ситуация интереснее, но тоже ничего нужного нам - ни расшифровки, ни каких-либо проверок. Просто чтение из класса и обращение к классу \verb|Cryptor|, который, судя по всему, нам и нужен. Декомпилируем и смотрим:
\begin{figure}[H]
\begin{center}
\includegraphics[width=0.7\linewidth]{files/java-cryptor}
\end{center}
\caption{Когда создал свою крипту}
\label{fig:java-classes}
\end{figure}

С первого взгляда флаг лежит прямо перед нами. Но это как-то слишком просто даже для easy-задачи. Посмотрим чуть ниже. Дейсвительно, сначала происходит какая-то проверка хэша. Если посмотреть внимательнее - никаких хешей нет. Сначала проверяем, что длина лицензии 17 символов, потом просто массив байтиков, с 9 по 15 элементы, сверяется с константой \verb|HASH_PATTERN|. После чего в функции \verb|decrypt| собирается флаг - обертка остается без изменений, а вот 7 символов ксорятся с \verb|HASH_PATTERN|. После чего совсем не сложно написать простенький скрипт для ксора или (что еще проще) написать скрипт, который "сгенерирует" лицензию и скормить ее программе:
\begin{lstlisting}[language=Python, caption=Генератор лицензии]
#!/usr/bin/env python3
# -*- coding: utf-8 -*-

def main():
    xored = ['\x00', '\x00', '\x00', '\x00', '\x00', '\x00', '\x00',
    '\x00', '\x00', '\x09', '\x15', '\x17', '\x0c', '\x10', '\x13', 
    '\x1c', '\x00']

    with open('license.bin', 'wb') as licensefile:
        for xb in xored:
            licensefile.write(bytes(xb, 'utf-8'))


if __name__ == "__main__":
    main()
\end{lstlisting}

И получаем флаг:
\begin{figure}[H]
\begin{center}
\includegraphics[width=0.7\linewidth]{files/java-flag}
\end{center}
\caption{Привет от Intel'a}
\label{fig:java-flag}
\end{figure}

%-----------------------------------------

\experiment{easy2}

\textbf{Теги:} C, several ways of solve\vspace{\baselineskip}

\begin{tcolorbox}
<условие задачи>
\end{tcolorbox}

Программа расшифровывает флаг и ждет от нас пароля, чтобы отдать его нам.
\begin{figure}[H]
\begin{center}
\includegraphics[width=0.5\linewidth]{files/enigma-hello}
\end{center}
\label{fig:enigma-hello}
\end{figure}

Посомтрим, что же в этот момент происходит внутри:
\begin{figure}[H]
\begin{center}
\includegraphics[width=1.0\linewidth]{files/enigma-asm}
\end{center}
\label{fig:enigma-asm}
\end{figure}

Глядя на этот листинг становится понятно, что действительно вызываются две функции. Судя по всему, одна из них для инициализации ключа, вторая для расшифровки. Таким образом, наш флга лежит в памяти еще до того, как программа спросила пароль. И тут появляется огромное количество возможных решений: к примеру, сдампить процесс, в дебаггере посмотреть содержимое кучи или, самый простой, - воспользоваться утилитой \verb|ltrace|, чтобы отследить все библиотечные вызовы - они тут есть, в этом можно убедиться, если посмотреть, что импортирует программа. Есть второй, более сложный путь решения, - увидеть, что пароль сравнивается с помощью функции \verb|strcmp| и поменять переход \verb|jnz| на \verb|jz| и, таким образом, при вводе неправильного пароля переходить на ветку, где программма печатает флаг. Ниже приведено решение с использованием \verb|ltrace|:
\begin{figure}[H]
\begin{center}
\includegraphics[width=1.0\linewidth]{files/enigma-ltrace}
\end{center}
\label{fig:enigma-asm}
\end{figure}

Как бонус, при решении через \verb|ltrace| можно также получить и пароль - это хорошо видно на скриншоте: \verb|reversegod|.

Отдельно стоит упомянуть решение "в лоб"{} - просто посидеть и прореверсить весь алгоритм. Это не так сложно - в данном случае был использован алгоритм, применявшийся в шифровальных машинах Энигма. Но для простой задачи это очень времязатратная операция, поэтому всегда стоит ставить в соответствие временные затраты и количество баллов, которые можно получить за задачу.

%----------------------------------------------------------------------------------------

\labday{Medium}

\experiment{medium}

\textbf{Теги:} <Теги>\vspace{\baselineskip}

\begin{tcolorbox}
<условие задачи>
\end{tcolorbox}

%----------------------------------------------------------------------------------------

\labday{Hard}

\experiment{hard1}

\textbf{Теги:} <Теги>\vspace{\baselineskip}

\begin{tcolorbox}
<условие задачи>
\end{tcolorbox}

%-----------------------------------------

\experiment{hard2}

\textbf{Теги:} <Теги>\vspace{\baselineskip}

\begin{tcolorbox}
<условие задачи>
\end{tcolorbox}

%----------------------------------------------------------------------------------------

\end{document}