%%%%%%%%%%%%%%%%%%%%%%%%%%%%%%%%%%%%%%%%%
% Daily Laboratory Book
% LaTeX Template
% Version 1.0 (4/4/12)
%
% This template has been downloaded from:
% http://www.LaTeXTemplates.com
%
% Original author:
% Frank Kuster (http://www.ctan.org/tex-archive/macros/latex/contrib/labbook/)
%
% Important note:
% This template requires the labbook.cls file to be in the same directory as the
% .tex file. The labbook.cls file provides the necessary structure to create the
% lab book.
%
% The \lipsum[#] commands throughout this template generate dummy text
% to fill the template out. These commands should all be removed when 
% writing lab book content.
%
% HOW TO USE THIS TEMPLATE 
% Each day in the lab consists of three main things:
%
% 1. LABDAY: The first thing to put is the \labday{} command with a date in 
% curly brackets, this will make a new page and put the date in big letters 
% at the top.
%
% 2. EXPERIMENT: Next you need to specify what experiment(s) you are 
% working on with an \experiment{} command with the experiment shorthand 
% in the curly brackets. The experiment shorthand is defined in the 
% 'DEFINITION OF EXPERIMENTS' section below, this means you can 
% say \experiment{pcr} and the actual text written to the PDF will be what 
% you set the 'pcr' experiment to be. If the experiment is a one off, you can 
% just write it in the bracket without creating a shorthand. Note: if you don't 
% want to have an experiment, just leave this out and it won't be printed.
%
% 3. CONTENT: Following the experiment is the content, i.e. what progress 
% you made on the experiment that day.
%
%%%%%%%%%%%%%%%%%%%%%%%%%%%%%%%%%%%%%%%%%

%----------------------------------------------------------------------------------------
%	PACKAGES AND OTHER DOCUMENT CONFIGURATIONS
%----------------------------------------------------------------------------------------

\documentclass[idxtotoc,hyperref,openany,oneside]{files/admin} % 'openany' here removes the gap page between days, erase it to restore this gap; 'oneside' can also be added to remove the shift that odd pages have to the right for easier reading

\usepackage[ 
  backref=page,
  pdfpagelabels=true,
  plainpages=false,
  colorlinks=true,
  bookmarks=true,
  pdfview=FitB]{hyperref} % Required for the hyperlinks within the PDF
  
\usepackage{booktabs} % Required for the top and bottom rules in the table
\usepackage{float} % Required for specifying the exact location of a figure or table
\usepackage{graphicx} % Required for including images2
\usepackage{listings} % Used for programs' listings
\usepackage{tcolorbox} % For textboxes
\usepackage{hyperref}

\usepackage[english,russian]{babel}
\usepackage[utf8]{inputenc}
\usepackage[T2A]{fontenc}

\newcommand{\HRule}{\rule{\linewidth}{0.5mm}} % Command to make the lines in the title page
\setlength\parindent{0pt} % Removes all indentation from paragraphs

%----------------------------------------------------------------------------------------
%	DEFINITION OF EXPERIMENTS
%----------------------------------------------------------------------------------------

\newexperiment{easy1}{Are you true admin?}
\newexperiment{easy2}{Do you like black terminals with green text?}
\newexperiment{medium1}{The best ID(A)E}
\newexperiment{medium2}{Broadcast}
\newexperiment{hard}{Just ping me}
\newexperiment{reallife}{Everybody want to be root}

%---------------------------------------------------------------------------------------

\begin{document}

%----------------------------------------------------------------------------------------
%	TITLE PAGE
%----------------------------------------------------------------------------------------

\frontmatter % Use Roman numerals for page numbers
\title{
\begin{center}
\HRule \\[0.4cm]
{\Huge \bfseries CTF Code \\[0.5cm] \Large Writeups}\\[0.4cm] % Degree
\HRule \\[1.5cm]
\end{center}
}
\author{\Huge Admin \\ \\[2cm]} % Your name and email address
\maketitle

\tableofcontents

\mainmatter % Use Arabic numerals for page numbers

%----------------------------------------------------------------------------------------
%	LAB BOOK CONTENTS
%----------------------------------------------------------------------------------------

% Blank template to use for new days:

%\labday{Day, Date Month Year}

%\experiment{}

%Text

%-----------------------------------------

%\experiment{}

%Text

%----------------------------------------------------------------------------------------

\labday{Easy}

\experiment{easy1}

\textbf{Теги:} Логи \vspace{\baselineskip}

\begin{tcolorbox}
<условие задачи>
\end{tcolorbox}

Нам даны два лога подключения: по FTP и по SSH соответственно. Если их открыть FTP лог, то пароль пользователя выглядит уж слишком похожим на base64. И действительно, если прогнать, то это выглядит как кусок флага. А после в SSH логе можно заметить, что ник похож на вторую часть флага в base64. Таким образом, просто два раза по base64 и получаем флаг \verb|oren_ctf_goto_ComRAT!|

%-----------------------------------------

\experiment{easy2}

\textbf{Теги:} Web, шифрование, brainfuck\vspace{\baselineskip}

\begin{tcolorbox}
<условие задачи>
\end{tcolorbox}

Нам дан сайт, который с первого взгляда абсолютно пуст. Но если посмотреть в develop-консоль, то можно увидеть, что сайт выдает туда некий base64. При расшифровке получаем:  \begin{verbatim}
Welcome aboard. Your secret phrase is "hyperion". 
Maybe you will need this: ----[-->+++++<]>.------.++++.--.+++[->+++<]>.[--->+<]>-.
++[->+++<]>.[--->+<]>--.+[->+++<]>.---------.[--->+<]>++.+++++.---.[------>+<]>.
[--->++<]>+++++.[->+++++<]>+++.[->+++<]>+.+++++.----.[----->++<]>+.+[----->+<]>.
++[---->+<]>++.|
\end{verbatim}

Последнее очень похоже на эзотерический язык программирования brainfuck. И дейсвительно, если выполнить эту программу любым интерпретатором, то можно получить \verb|vptr_tbt_VtyviKzotpaz!|, что оооочень напоминает по формату флаг. Остается вспомнить, какие шифры требуют ключ. Первым, как самым популярным, на ум приходит шифр Виженера. И действительно, если прогнать через любой расшифровщик, то получаем \verb|oren_ctf_ImageTragick!|

%----------------------------------------------------------------------------------------

\labday{Medium}

\experiment{medium1}

\textbf{Теги:} User-agent, SHA1, requests\vspace{\baselineskip}

\begin{tcolorbox}
<условие задачи>
\end{tcolorbox}

При переходе по ссылке внезапно оказывается, что сайт не любит гостей. Но если посмотреть на заголовок ответа, то можно увидеть, что нам предлагают перейти по какой-то ссылке. Похоже на какой-то хеш. Если посчитать количество символов, то можно понять, что это SHA1, символов ровно 20 штук. Брут ничего не дает, поэтому можно просто перейти. Опять предложение перейти по ссылке и уже ответ, что мы забанены. Тогда можно написать простой питоновский скрипт, который просто ходит по ссылкам. В конце концов, судя по названия таска, где-то в конце будет флаг
\begin{lstlisting}[language=Python]
from urllib.request import urlopen, Request
import sys


def get_next_message(hash_str, const):
    new_request = Request(
        url="http://{}/".format(sys.argv[1]) + hash_str,
        headers={'User-Agent': "snake3.smth + {}".format(const)}
    )
    print(hash_str)
    return urlopen(new_request).msg


i = 0
global_hash_string = get_next_message("", i)
while "oren_ctf" not in global_hash_string:
    i += 1
    print(i, global_hash_string)
    global_hash_string = get_next_message(global_hash_string.split()[-1], i)
print(global_hash_string)
\end{lstlisting}

И действительно, в конце концов получаем флаг \verb|oren_ctf_DUHK!|

%-----------------------------------------

\experiment{medium2}

\textbf{Теги:} \vspace{\baselineskip}

\begin{tcolorbox}
<условие задачи>
\end{tcolorbox}

%----------------------------------------------------------------------------------------

\labday{Hard}

\experiment{hard}

\textbf{Теги:} ICMP, Network analysis\vspace{\baselineskip}

\begin{tcolorbox}
<условие задачи>
\end{tcolorbox}

Дан единственный ICMP-запрос на хост. Но он выглядит не как стандартный ICMP - в секции данных что-то еще спрятано. Первое желение - послать такой же запрос на хост. И дейсвительно, в ответ прилетает флаг \verb|oren_ctf_Rowhammer!|

%----------------------------------------------------------------------------------------

\labday{Real life}

\experiment{reallife}

\textbf{Теги:} \vspace{\baselineskip}

\begin{tcolorbox}
<условие задачи>
\end{tcolorbox}

%----------------------------------------------------------------------------------------

\end{document}